% Options for packages loaded elsewhere
\PassOptionsToPackage{unicode}{hyperref}
\PassOptionsToPackage{hyphens}{url}
\PassOptionsToPackage{dvipsnames,svgnames,x11names}{xcolor}
%
\documentclass[
  letterpaper,
  DIV=11,
  numbers=noendperiod]{scrartcl}

\usepackage{amsmath,amssymb}
\usepackage{setspace}
\usepackage{iftex}
\ifPDFTeX
  \usepackage[T1]{fontenc}
  \usepackage[utf8]{inputenc}
  \usepackage{textcomp} % provide euro and other symbols
\else % if luatex or xetex
  \usepackage{unicode-math}
  \defaultfontfeatures{Scale=MatchLowercase}
  \defaultfontfeatures[\rmfamily]{Ligatures=TeX,Scale=1}
\fi
\usepackage{lmodern}
\ifPDFTeX\else  
    % xetex/luatex font selection
\fi
% Use upquote if available, for straight quotes in verbatim environments
\IfFileExists{upquote.sty}{\usepackage{upquote}}{}
\IfFileExists{microtype.sty}{% use microtype if available
  \usepackage[]{microtype}
  \UseMicrotypeSet[protrusion]{basicmath} % disable protrusion for tt fonts
}{}
\makeatletter
\@ifundefined{KOMAClassName}{% if non-KOMA class
  \IfFileExists{parskip.sty}{%
    \usepackage{parskip}
  }{% else
    \setlength{\parindent}{0pt}
    \setlength{\parskip}{6pt plus 2pt minus 1pt}}
}{% if KOMA class
  \KOMAoptions{parskip=half}}
\makeatother
\usepackage{xcolor}
\usepackage[left=2.54cm,right=2.54cm,top=2.54cm,bottom=2.54cm]{geometry}
\setlength{\emergencystretch}{3em} % prevent overfull lines
\setcounter{secnumdepth}{-\maxdimen} % remove section numbering
% Make \paragraph and \subparagraph free-standing
\makeatletter
\ifx\paragraph\undefined\else
  \let\oldparagraph\paragraph
  \renewcommand{\paragraph}{
    \@ifstar
      \xxxParagraphStar
      \xxxParagraphNoStar
  }
  \newcommand{\xxxParagraphStar}[1]{\oldparagraph*{#1}\mbox{}}
  \newcommand{\xxxParagraphNoStar}[1]{\oldparagraph{#1}\mbox{}}
\fi
\ifx\subparagraph\undefined\else
  \let\oldsubparagraph\subparagraph
  \renewcommand{\subparagraph}{
    \@ifstar
      \xxxSubParagraphStar
      \xxxSubParagraphNoStar
  }
  \newcommand{\xxxSubParagraphStar}[1]{\oldsubparagraph*{#1}\mbox{}}
  \newcommand{\xxxSubParagraphNoStar}[1]{\oldsubparagraph{#1}\mbox{}}
\fi
\makeatother


\providecommand{\tightlist}{%
  \setlength{\itemsep}{0pt}\setlength{\parskip}{0pt}}\usepackage{longtable,booktabs,array}
\usepackage{calc} % for calculating minipage widths
% Correct order of tables after \paragraph or \subparagraph
\usepackage{etoolbox}
\makeatletter
\patchcmd\longtable{\par}{\if@noskipsec\mbox{}\fi\par}{}{}
\makeatother
% Allow footnotes in longtable head/foot
\IfFileExists{footnotehyper.sty}{\usepackage{footnotehyper}}{\usepackage{footnote}}
\makesavenoteenv{longtable}
\usepackage{graphicx}
\makeatletter
\newsavebox\pandoc@box
\newcommand*\pandocbounded[1]{% scales image to fit in text height/width
  \sbox\pandoc@box{#1}%
  \Gscale@div\@tempa{\textheight}{\dimexpr\ht\pandoc@box+\dp\pandoc@box\relax}%
  \Gscale@div\@tempb{\linewidth}{\wd\pandoc@box}%
  \ifdim\@tempb\p@<\@tempa\p@\let\@tempa\@tempb\fi% select the smaller of both
  \ifdim\@tempa\p@<\p@\scalebox{\@tempa}{\usebox\pandoc@box}%
  \else\usebox{\pandoc@box}%
  \fi%
}
% Set default figure placement to htbp
\def\fps@figure{htbp}
\makeatother
% definitions for citeproc citations
\NewDocumentCommand\citeproctext{}{}
\NewDocumentCommand\citeproc{mm}{%
  \begingroup\def\citeproctext{#2}\cite{#1}\endgroup}
\makeatletter
 % allow citations to break across lines
 \let\@cite@ofmt\@firstofone
 % avoid brackets around text for \cite:
 \def\@biblabel#1{}
 \def\@cite#1#2{{#1\if@tempswa , #2\fi}}
\makeatother
\newlength{\cslhangindent}
\setlength{\cslhangindent}{1.5em}
\newlength{\csllabelwidth}
\setlength{\csllabelwidth}{3em}
\newenvironment{CSLReferences}[2] % #1 hanging-indent, #2 entry-spacing
 {\begin{list}{}{%
  \setlength{\itemindent}{0pt}
  \setlength{\leftmargin}{0pt}
  \setlength{\parsep}{0pt}
  % turn on hanging indent if param 1 is 1
  \ifodd #1
   \setlength{\leftmargin}{\cslhangindent}
   \setlength{\itemindent}{-1\cslhangindent}
  \fi
  % set entry spacing
  \setlength{\itemsep}{#2\baselineskip}}}
 {\end{list}}
\usepackage{calc}
\newcommand{\CSLBlock}[1]{\hfill\break\parbox[t]{\linewidth}{\strut\ignorespaces#1\strut}}
\newcommand{\CSLLeftMargin}[1]{\parbox[t]{\csllabelwidth}{\strut#1\strut}}
\newcommand{\CSLRightInline}[1]{\parbox[t]{\linewidth - \csllabelwidth}{\strut#1\strut}}
\newcommand{\CSLIndent}[1]{\hspace{\cslhangindent}#1}

\KOMAoption{captions}{tableheading}
\makeatletter
\@ifpackageloaded{caption}{}{\usepackage{caption}}
\AtBeginDocument{%
\ifdefined\contentsname
  \renewcommand*\contentsname{Tabla de contenidos}
\else
  \newcommand\contentsname{Tabla de contenidos}
\fi
\ifdefined\listfigurename
  \renewcommand*\listfigurename{Listado de Figuras}
\else
  \newcommand\listfigurename{Listado de Figuras}
\fi
\ifdefined\listtablename
  \renewcommand*\listtablename{Listado de Tablas}
\else
  \newcommand\listtablename{Listado de Tablas}
\fi
\ifdefined\figurename
  \renewcommand*\figurename{Figura}
\else
  \newcommand\figurename{Figura}
\fi
\ifdefined\tablename
  \renewcommand*\tablename{Tabla}
\else
  \newcommand\tablename{Tabla}
\fi
}
\@ifpackageloaded{float}{}{\usepackage{float}}
\floatstyle{ruled}
\@ifundefined{c@chapter}{\newfloat{codelisting}{h}{lop}}{\newfloat{codelisting}{h}{lop}[chapter]}
\floatname{codelisting}{Listado}
\newcommand*\listoflistings{\listof{codelisting}{Listado de Listados}}
\makeatother
\makeatletter
\makeatother
\makeatletter
\@ifpackageloaded{caption}{}{\usepackage{caption}}
\@ifpackageloaded{subcaption}{}{\usepackage{subcaption}}
\makeatother

\ifLuaTeX
\usepackage[bidi=basic]{babel}
\else
\usepackage[bidi=default]{babel}
\fi
\babelprovide[main,import]{spanish}
% get rid of language-specific shorthands (see #6817):
\let\LanguageShortHands\languageshorthands
\def\languageshorthands#1{}
\usepackage{bookmark}

\IfFileExists{xurl.sty}{\usepackage{xurl}}{} % add URL line breaks if available
\urlstyle{same} % disable monospaced font for URLs
\hypersetup{
  pdftitle={Perfiles de Individualismo en la sociedad chilena},
  pdfauthor={Gabriel Cortés Paredes},
  pdflang={es},
  pdfkeywords={Individualismo, Individuación, Análisis de clases
latentes},
  colorlinks=true,
  linkcolor={blue},
  filecolor={Maroon},
  citecolor={Blue},
  urlcolor={Blue},
  pdfcreator={LaTeX via pandoc}}


\title{Perfiles de Individualismo en la sociedad chilena}
\author{Gabriel Cortés Paredes}
\date{}

\begin{document}
\maketitle
\begin{abstract}
Escribir
\end{abstract}


\setstretch{1.15}
\section{Introducción}\label{introducciuxf3n}

Desde los inicios de las ciencias sociales, la pregunta por cómo las
sociedades se mantienen unidas ha sido central para la reflexión
sociológica. Aunque la modernidad se asoció con una mayor autonomía
individual, el individualismo fue visto con sospecha por figuras
fundacionales. Si bien este nuevo fenómeno estuvo lejos de significar el
fin de la sociedad, autores como Alexis de Tocqueville y Émile Durkheim
advirtieron que un individualismo extremo puede derivar en patologías
sociales, como la fragmentación de los vínculos comunitarios.

Esta preocupación adquiere especial relevancia en el contexto
contemporáneo, marcado por una acumulación de crisis sociales, políticas
y económicas que tensionan la cohesión social a escala global. En Chile,
la cuestión se vuelve particularmente visible tras el ciclo abierto por
la crisis social de 2019, las frustradas experiencias constituyentes y
crisis sucesivas ---sanitaria, migratoria y de seguridad---, que han
reactivado el debate sobre los lazos sociales y cívicos {[}citar cosas
cepal, coes y mideso{]}

Chile se ha caracterizado por el avance de reformas neoliberales
profundas durante la dictadura militar (1973--1990) y su posterior
consolidación en democracia. Dichas transformaciones excedieron el plano
económico e instalaron un discurso hegemónico que enfatiza el esfuerzo
personal, la competencia y la meritocracia como vías legítimas de logro.
Políticas públicas en educación, salud y bienestar reforzaron la
responsabilidad individual, favoreciendo subjetividades centradas en la
agencia personal (\citeproc{ref-araujo2012}{Araujo y Martuccelli 2012}),
legitimando desigualdades económicas
(\citeproc{ref-castillo2024}{\textbf{castillo2024?}}), favoreciendo el
actuar estratégico (\citeproc{ref-araujo2014}{Araujo y Martuccelli
2014}) y, eventualmente, inhibiendo el cambio político
(\citeproc{ref-pnud2024}{\textbf{pnud2024?}}).

Desde esta perspectiva, la presencia de un~\emph{individualismo asocial}
(\citeproc{ref-pnud2024}{\textbf{pnud2024?}}) operaría como fuerza
disgregadora de lazos sociales y ciudadanos, configurándose como un
desafío para la mantención de la cohesión social en Chile. En este
contexto, este artículo busca aportar a la discusión empírica
identificando los distintos~\textbf{perfiles de
individualismo}~presentes en la sociedad chilena.

El estudio del individualismo ha estado dominado por la psicología
cultural, especialmente a partir del enfoque popularizado por Geert
Hofstede en los años ochenta. En su formulación, el individualismo y el
colectivismo constituyen polos de un continuo unidimensional que
permitiría distinguir entre culturas individualistas y colectivistas
(\citeproc{ref-oyserman2002}{Oyserman, Coon, y Kemmelmeier 2002};
\citeproc{ref-yoon2010}{Yoon 2010}). Las primeras se caracterizan por
lazos poco estrechos y expectativas de autosuficiencia individual y
familiar; las segundas, por la integración temprana en grupos
cohesionados que brindan protección a cambio de lealtad
(\citeproc{ref-yoon2010}{Yoon 2010}).

El fenómeno del individualismo ha sido abordado principalmente desde la
psicología cultural, particularmente desde el enfoque popularizado por
Geert Hofstede en la década de 1980. Para Hofstede, el individualismo es
el polo de un espectro continuo y unidimensional que tiene en su otro
extremo al colectivismo. De tal modo, sería posible distinguir entre
culturas individualistas y culturas colectivistas
(\citeproc{ref-oyserman2002}{Oyserman, Coon, y Kemmelmeier 2002};
\citeproc{ref-yoon2010}{Yoon 2010}). Las sociedades individualistas se
caracterizarían por la existencia de lazos poco estrechos entre sus
individuos, de quienes se espera se hagan cargos de sí mismos y de su
familia inmediata. Las sociedades colectivistas, en tanto, se definen
porque sus miembros están integrados desde su nacimiento a grupos
fuertemente cohesionados que los protegen a lo largo de sus vidas a
cambio de una lealtad incuestionable (\citeproc{ref-yoon2010}{Yoon
2010}).

Pese a su influencia, este enfoque ha recibido críticas por su vaguedad
conceptual ---a veces tratado como un ``catch-all'' para explicar toda
diferencia cultural (\citeproc{ref-voronov2002}{Voronov y Singer
2002})--- y por su sesgo normativo, que asocia el individualismo con la
modernidad y el desarrollo (\citeproc{ref-martuccelli2010}{Martuccelli
2010}; \citeproc{ref-voronov2002}{Voronov y Singer 2002};
\citeproc{ref-wang2010}{Wang y Liu 2010}). También se le objeta la
imprecisión en la definición de ``colectivos'' ---sin distinguir con
claridad entre grupos, colectivos y comunidades
(\citeproc{ref-brewer2007}{Brewer y Chen 2007};
\citeproc{ref-moemeka1998}{Moemeka 1998};
\citeproc{ref-oyserman2002}{Oyserman, Coon, y Kemmelmeier 2002})--- y la
frecuente confusión entre niveles de análisis (cultural vs.~individual),
a menudo solapando cofundiéndose a nivel operacional y teórico con
conceptos como el \emph{self-construal} (\citeproc{ref-cross2011}{Cross,
Hardin, y Gercek-Swing 2011}; \citeproc{ref-voronov2002}{Voronov y
Singer 2002}).

A ello se suman problemas de operacionalización
(\citeproc{ref-brewer2007}{Brewer y Chen 2007};
\citeproc{ref-oyserman2002}{Oyserman, Coon, y Kemmelmeier 2002}). Brewer
y Chen (\citeproc{ref-brewer2007}{2007}) señalan la existencia de una
asimetría habitual en las operacionalizaciones de los conceptos:
mientras el individualismo se mide con ítems sobre identidad y agencia
personal, el colectivismo suele capturarse como un sistema de valores.

Estas discrepancias conceptuales podrían explicar las anomalías
observados en varios de estos estudios, como que los individualistas
pueden ser tanto o más colectivistas que colectivistas mismos
(\citeproc{ref-oyserman2002}{Oyserman, Coon, y Kemmelmeier 2002}), o que
en determinados contextos los colectivistas actúan de manera
individualista (\citeproc{ref-voronov2002}{Voronov y Singer 2002}).

En el plano agregado, Chile ilustra esas tensiones. Bajo la definición
de Hofstede, la sociedad chilena ha sido clasificada como colectivista
(\citeproc{ref-leonquillas2022}{\textbf{leonquillas2022?}};
\citeproc{ref-rojas2008}{Rojas-Méndez et~al. 2008}). Esto concuerda con
hallazgos que reportan altos niveles de colectivismo, sea como opuesto
al individualismo (\citeproc{ref-oyserman2002}{Oyserman, Coon, y
Kemmelmeier 2002}) o como self-construal interdependiente
(\citeproc{ref-benavides2020}{Benavides y Hur 2020}). Sin embargo, otras
mediciones muestran niveles de individualismo en Chile que son
comparables o superiores a los de sociedades típicamente
individualistas, como Estados Unidos
(\citeproc{ref-oyserman2002}{Oyserman, Coon, y Kemmelmeier 2002}) o
Noruega (\citeproc{ref-kolstad2009}{Kolstad y Horpestad 2009}).

Surge así una doble pregunta: ¿es realmente una sociedad colectivista?,
y si no lo es, ¿hasta qué punto es una sociedad individualista? La
propuesta de esta investigación es que, con el fin de responder esta
pregunta, es necesario dar un giro hacia una perspectiva teórica que
provea el lenguaje para describir el individualismo chileno.
Particularmente, este artículo buscará responder esta pregunta a través
del lente de la teoría de la individualización.

Desde fines de los noventa, la teoría de la individualización ha sido un
marco ampliamente utilizado en las ciencias sociales para analizar
transformaciones culturales, sociales y económicas
(\citeproc{ref-yopo2013}{Yopo 2013}), aplicándose a temáticas como
género y familia (\citeproc{ref-murray2022}{\textbf{murray2022?}}),
religión (\citeproc{ref-baeza2022}{\textbf{baeza2022?}};
\citeproc{ref-cortesparedes2022}{\textbf{cortesparedes2022?}}), trabajo
(\citeproc{ref-soto2021}{\textbf{soto2021?}}), seguridad
(\citeproc{ref-trebilcock2019}{\textbf{trebilcock2019?}}), discapacidad
(\citeproc{ref-solsona-cisternas2023}{\textbf{solsona-cisternas2023?}})
y educación
(\citeproc{ref-canalesceron2021}{\textbf{canalesceron2021?}};
\citeproc{ref-pinheiro2023}{\textbf{pinheiro2023?}}). No obstante, se ha
advertido su adopción a veces acrítica, con escasa atención a
especificidades nacionales y latinoamericanas y con limitada
incorporación de variables sociodemográficas que permitan identificar
diferencias entre grupos (\citeproc{ref-gayo2017}{\textbf{gayo2017?}};
\citeproc{ref-yopo2013}{Yopo 2013}).

Sin embargo, se ha advertido que muchas de las investigaciones que
siguen esta línea teórica lo hacen a través de una aproximación acrítica
a las claves interpretativas de la teoría, sin llegar a dar cuenta de
las particularidades de estos procesos en Chile y en América Latina. Por
otro lado, se ha observado que en los estudios que utilizan esta
perspectiva teórica rara vez se usan variables sociodemográficas de
manera explicativa, descuidando posibles diferencias entre grupos
sociales (\citeproc{ref-gayo2017}{\textbf{gayo2017?}}).

Reconocer estas limitaciones es crucial, ya que existe el riesgo de
asumir una individuación homogénea dentro de la sociedad, sin
contrastación empírica suficiente, ya sea por imprecisiones conceptual
(\citeproc{ref-yopo2013}{Yopo 2013}) o por déficits metodológicos
(\citeproc{ref-gayo2017}{\textbf{gayo2017?}}).Esta brecha existe pese al
consenso aparente de que la individuación es un proceso que diverge no
solo entre culturas, sino también dentro de una misma sociedad
(\citeproc{ref-martuccelli2018}{Martuccelli 2018}).

Ante esto, el siguiente artículo busca abordar estas diferencias no solo
de manera declarativa a nivel teórico, sino identificarlas
empíricamente. A continuación, se presenta el marco conceptual y las
definiciones centrales de esta investigación; luego, al estrategia
metodológica; posteriormente, los hallazgos, caracterizando los perfiles
identificados. Finalmente, se introduce discuten los resultados a la luz
del modelo teórico y las limitaciones del estudio, junto con
proyecciones para futuras investigaciones.

\hfill\break

\phantomsection\label{refs}
\begin{CSLReferences}{1}{0}
\bibitem[\citeproctext]{ref-araujo2012}
Araujo, Kathya, y Danilo Martuccelli. 2012. \emph{Desaf{í}os {Comunes}.
{Retrato} de La Sociedad Chilena y Sus Individuos}. LOM.

\bibitem[\citeproctext]{ref-araujo2014}
---------. 2014. {«Beyond Institutional Individualism: {Agentic}
Individualism and the Individuation Process in {Chilean} Society»}.
\emph{Current Sociology} 62 (1): 24-40.
\url{https://doi.org/10.1177/0011392113512496}.

\bibitem[\citeproctext]{ref-benavides2020}
Benavides, Paloma, y Taekyun Hur. 2020. {«Self-{Construal Differences}
in {Chile} and {South Korea}: {A Brief Report}»}. \emph{Psychological
Reports} 123 (6): 2410-17.
\url{https://doi.org/10.1177/0033294119868786}.

\bibitem[\citeproctext]{ref-brewer2007}
Brewer, Marilynn B., y Ya-Ru Chen. 2007. {«Where ({Who}) {Are
Collectives} in {Collectivism}? {Toward Conceptual Clarification} of
{Individualism} and {Collectivism}.»} \emph{Psychological Review} 114
(1): 133-51. \url{https://doi.org/10.1037/0033-295X.114.1.133}.

\bibitem[\citeproctext]{ref-cross2011}
Cross, Susan E., Erin E. Hardin, y Berna Gercek-Swing. 2011. {«The
{\emph{What}}{\emph{,} }{\emph{How}}{\emph{,} }{\emph{Why}}{\emph{, and}
}{\emph{Where}} of {Self-Construal}»}. \emph{Personality and Social
Psychology Review} 15 (2): 142-79.
\url{https://doi.org/10.1177/1088868310373752}.

\bibitem[\citeproctext]{ref-kolstad2009}
Kolstad, Arnulf, y Silje Horpestad. 2009. {«Self-{Construal} in {Chile}
and {Norway}: {Implications} for {Cultural Differences} in
{Individualism} and {Collectivism}»}. \emph{Journal of Cross-Cultural
Psychology} 40 (2): 275-81.
\url{https://doi.org/10.1177/0022022108328917}.

\bibitem[\citeproctext]{ref-martuccelli2010}
Martuccelli, Danilo. 2010. \emph{{{\textquestiondown}Existen individuos
en el sur?}} Santiago de Chile: LOM.

\bibitem[\citeproctext]{ref-martuccelli2018}
---------. 2018. {«Variantes Del Individualismo»}. \emph{Estudios
Sociol{ó}gicos de El Colegio de M{é}xico} 37 (109): 7-37.
\url{https://doi.org/10.24201/es.2019v37n109.1732}.

\bibitem[\citeproctext]{ref-moemeka1998}
Moemeka, Andrew A. 1998. {«Communalism as a {Fundamental Dimension} of
{Culture}»}. \emph{Journal of Communication} 48 (4): 118-41.
\url{https://doi.org/10.1111/j.1460-2466.1998.tb02773.x}.

\bibitem[\citeproctext]{ref-oyserman2002}
Oyserman, Daphna, Heather M. Coon, y Markus Kemmelmeier. 2002.
{«Rethinking Individualism and Collectivism: {Evaluation} of Theoretical
Assumptions and Meta-Analyses.»} \emph{Psychological Bulletin} 128 (1):
3-72. \url{https://doi.org/10.1037/0033-2909.128.1.3}.

\bibitem[\citeproctext]{ref-rojas2008}
Rojas-Méndez, José I., Vilma Coutiño-Hill, Rabi S. Bhagat, y Karen South
Moustafa. 2008. {«{Evaluaci{ó}n del individualismo y colectivismo
horizontal y vertical en la sociedad Chilena.}»} \emph{Multidisciplinary
Business Review} 1 (1): 36-48.

\bibitem[\citeproctext]{ref-voronov2002}
Voronov, Maxim, y Jefferson A. Singer. 2002. {«The {Myth} of
{Individualism-Collectivism}: {A Critical Review}»}. \emph{The Journal
of Social Psychology} 142 (4): 461-80.
\url{https://doi.org/10.1080/00224540209603912}.

\bibitem[\citeproctext]{ref-wang2010}
Wang, Georgette, y Zhong-Bo Liu. 2010. {«What Collective? {Collectivism}
and Relationalism from a {Chinese} Perspective»}. \emph{Chinese Journal
of Communication} 3 (1): 42-63.
\url{https://doi.org/10.1080/17544750903528799}.

\bibitem[\citeproctext]{ref-yoon2010}
Yoon, Kwang-Il. 2010. {«Political {Culture} of {Individualism} and
{Collectivism}»}. A Dissertation for the Degree of \{\{Doctor\}\} of
\{\{Philosophy\}\} (\{\{Political Science\}\}), Universidad de Michigan.

\bibitem[\citeproctext]{ref-yopo2013}
Yopo, Martina. 2013. {«{Individualizaci{ó}n en Chile. Individuo y
sociedad en las transformaciones culturales recientes}»}.
\emph{Psicoperspectivas. Individuo y Sociedad} 12 (2): 4-15.
\url{https://doi.org/10.5027/psicoperspectivas-Vol12-Issue2-fulltext-254}.

\end{CSLReferences}




\end{document}
